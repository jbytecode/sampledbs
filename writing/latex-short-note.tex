\documentclass[12pt]{article}
\usepackage[utf8]{inputenc}
\usepackage{amsmath}
\usepackage{amsfonts}
\usepackage{amssymb}
\usepackage{lipsum}
\usepackage{xcolor}
\usepackage{hyperref}

\title{A Brief Note on \LaTeX}
\author{Mehmet Hakan Satman}
\date{\today}

\begin{document}
\maketitle
\tableofcontents

\section{documentclass}

\begin{verbatim}
\documentclass[12pt]{article}
\end{verbatim}

\section{Packages}
\begin{verbatim}
\usepackage[utf8]{inputenc}
\usepackage{amsmath}
\usepackage{amsfonts}
\usepackage{amssymb}
\usepackage{lipsum}
\usepackage{xcolor}
\usepackage{hyperref}
\end{verbatim}  


\section{Meta information}

\subsection{Title}
\begin{verbatim}
\title{A Brief Note on \LaTeX}
\end{verbatim}

\subsection{Author}
\begin{verbatim}
\author{Mehmet Hakan Satman}
\end{verbatim}

\subsection{Date}
\begin{verbatim}
\date{\today}
\end{verbatim}

\subsection{Making Title}
\begin{verbatim}
\maketitle
\end{verbatim}

\section{Table of Contents}
\begin{verbatim}
\tableofcontents
\end{verbatim}


\section{Paragraphs}

This is the first paragraph. It contains some text to 
demonstrate how paragraphs are created in \LaTeX.

This is the second paragraph. Notice the space between 
this paragraph and the previous one, which indicates a new paragraph has started.

\noindent This is the third paragraph. It follows immediately 
after the second paragraph without any additional vertical space.

\begin{verbatim}
This is the first paragraph. It contains some text to 
demonstrate how paragraphs are created in \LaTeX.

This is the second paragraph. Notice the space between
 this paragraph and the previous one, which indicates
  a new paragraph has started.

\noindent This is the third paragraph. It follows immediately
 after the second paragraph without any additional vertical space.
\end{verbatim}


\section{Centering}
\begin{center}
This text is centered in the document.
\end{center}

\begin{verbatim}
\begin{center}
This text is centered in the document.
\end{center}
\end{verbatim}



\section{Quotes}

Inline Quote: ``This is an inline quote in \LaTeX.''

\begin{verbatim}
``This is an inline quote in \LaTeX.''
\end{verbatim}


\section{Bold, italic, and underline Text}

Bold Text: \textbf{This text is bold.}
\begin{verbatim}
\textbf{This text is bold.}
\end{verbatim}

Italic Text: \textit{This text is italic.}
\begin{verbatim}
\textit{This text is italic.}
\end{verbatim}

Underline Text: \underline{This text is underlined.}
\begin{verbatim}
\underline{This text is underlined.}
\end{verbatim}

Bold and Italic Text: \textbf{\textit{This text is bold and italic.}}
\begin{verbatim}
\textbf{\textit{This text is bold and italic.}}
\end{verbatim}

Bold and Underline Text: \textbf{\underline{This text is bold and underlined.}}
\begin{verbatim}
\textbf{\underline{This text is bold and underlined.}}
\end{verbatim}  

Bold, italic, and Underline Text: \textbf{\textit{\underline{This text is bold, italic, and underlined.}}}
\begin{verbatim}
\textbf{\textit{\underline{This text is bold, italic, and underlined.}}}
\end{verbatim}  


\section{Colored Text}

Required package: \verb|\usepackage{xcolor}|

Red : \textcolor{red}{This text is red.}
\begin{verbatim}
\textcolor{red}{This text is red.}
\end{verbatim}

Blue: \textcolor{blue}{This text is blue.}
\begin{verbatim}    
\textcolor{blue}{This text is blue.}
\end{verbatim}

Green: \textcolor{green}{This text is green.}
\begin{verbatim}
\textcolor{green}{This text is green.}
\end{verbatim}

Orange: \textcolor{orange}{This text is orange.}
\begin{verbatim}
\textcolor{orange}{This text is orange.}
\end{verbatim}


\section{URLs}

Required package: \verb|\usepackage{hyperref}|

\url{https://www.example.com}

\begin{verbatim}
\url{https://www.example.com}
\end{verbatim}




\section{Sectioning Commands}

Section:

\begin{verbatim}
\section{Section Title}
\end{verbatim}

Subsection:

\begin{verbatim}
\subsection{Subsection Title}
\end{verbatim}

Subsubsection:

\begin{verbatim}
\subsubsection{Subsubsection Title}
\end{verbatim}



\section{Mathematical Expressions}


$$
E = mc^2
$$

\begin{verbatim}
$$
E = mc^2
$$
\end{verbatim}


\hrule

$$
\int_a^b f(x) \, dx = F(b) - F(a)
$$

\begin{verbatim}
$$
\int_a^b f(x) \, dx = F(b) - F(a)
$$
\end{verbatim}

\hrule

$$
\Delta = b^2 - 4ac
$$

\begin{verbatim}
$$
\Delta = b^2 - 4ac
$$
\end{verbatim}

\hrule

$$
\int_0^\infty \frac{1}{\theta} e^{-\frac{x}{\theta}} \, dx = 1
$$

\begin{verbatim}
$$
\int_0^\infty \frac{1}{\theta} e^{-\frac{x}{\theta}} \, dx = 1
$$
\end{verbatim}

\hrule

$$
p(x) = a_n x^n + a_{n-1} x^{n-1} + \ldots + a_1 x + a_0
$$

\begin{verbatim}
$$
p(x) = a_n x^n + a_{n-1} x^{n-1} + \ldots + a_1 x + a_0
$$
\end{verbatim}

\hrule

$$
f(x) = \sum_{n=0}^{\infty} \frac{f^{(n)}(a)}{n!} (x - a)^n
$$

\begin{verbatim}
$$
f(x) = \sum_{n=0}^{\infty} \frac{f^{(n)}(a)}{n!} (x - a)^n
$$
\end{verbatim}

\hrule

$$
f(x) = \begin{cases} 
      x^2 & x \geq 0 \\
      -x & x < 0 
   \end{cases}
$$

\begin{verbatim}
$$
f(x) = \begin{cases} 
      x^2 & x \geq 0 \\
      -x & x < 0 
   \end{cases}
$$
\end{verbatim}

\hrule

$$
sign(x) = \begin{cases} 
      1 & x > 0 \\
      0 & x = 0 \\
      -1 & x < 0 
   \end{cases}
$$

\begin{verbatim}
$$
sign(x) = \begin{cases} 
      1 & x > 0 \\
      0 & x = 0 \\
      -1 & x < 0 
   \end{cases}
$$
\end{verbatim}  

\hrule

$$
abs(x) = \begin{cases} 
      x & x \geq 0 \\
      -x & x < 0 
   \end{cases}  
$$

\begin{verbatim}
$$
abs(x) = \begin{cases} 
      x & x \geq 0 \\
      -x & x < 0 
   \end{cases}  
$$
\end{verbatim}

\hrule


$$
A = \begin{bmatrix}
1 & 2 & 4 \\
0 & 1 & 3 \\
0 & 0 & 1
\end{bmatrix}
$$

\begin{verbatim}
$$
A = \begin{bmatrix}
1 & 2 & 4 \\
0 & 1 & 3 \\
0 & 0 & 1
\end{bmatrix}
$$
\end{verbatim}

\hrule 

\begin{equation}
\begin{split}
    \max z = & x + 2y \\
    \text{Subject to:} &  \\
    & 2x + 3y \leq 52 \\
    & x + y \ge 2 \\
    & x \geq 0, y \geq 0 \\
\end{split}
\end{equation}


\begin{verbatim}
\begin{equation}
\begin{split}
    \max z = & x + 2y \\
    \text{Subject to:} &  \\
    & 2x + 3y \leq 52 \\
    & x + y \ge 2 \\
    & x \geq 0, y \geq 0 \\
\end{split}
\end{equation}
\end{verbatim}

\hrule 

\begin{table}[h!]
\centering
\begin{tabular}{|c|c|}
\hline
Symbol & Description \\
\hline
$\alpha$ & \verb|\alpha| \\
$\beta$ & \verb|\beta| \\
$\gamma$ & \verb|\gamma| \\
$\delta$ & \verb|\delta| \\
$\Delta$ & \verb|\Delta| \\
$\epsilon$ & \verb|\epsilon| \\
$\varepsilon$ & \verb|\varepsilon| \\
$\gamma$ & \verb|\gamma| \\
$\Gamma$ & \verb|\Gamma| \\
$\rho$ & \verb|\rho| \\
$\eta$ & \verb|\eta| \\
$\zeta$ & \verb|\zeta| \\
$\theta$ & \verb|\theta| \\
$\lambda$ & \verb|\lambda| \\
$\mu$ & \verb|\mu| \\
$\sigma$ & \verb|\sigma| \\
$\pi$ & \verb|\pi| \\
$\Sigma$ & \verb|\Sigma| \\
$\int$ & \verb|\int| \\
$\sum$ & \verb|\sum| \\
$\frac{a}{b}$ & \verb|\frac{a}{b}| \\
$\sqrt{x}$ & \verb|\sqrt{x}| \\
$\sqrt[n]{x}$ & \verb|\sqrt[n]{x}| \\
$\infty$ & \verb|\infty| \\
$x_i$ & \verb|x_i| \\
$x^n$ & \verb|x^n| \\
$x_{i}^{n}$ & \verb|x_{i}^{n}| \\
\hline
\end{tabular}
\caption{Common Mathematical Symbols in \LaTeX}
\end{table}







\section{Lorem Ipsum}

\lipsum[1-2]

\begin{verbatim}
\lipsum[1-2]
\end{verbatim}

\section{Tables}
\subsection{Tabular Environment}
\begin{tabular}{|c|c|c|}
\hline
A & B & C \\
\hline
1 & 2 & 3 \\
4 & 5 & 6 \\
\hline
\end{tabular}

\begin{verbatim}
\begin{tabular}{|c|c|c|}
\hline
A & B & C \\
\hline
1 & 2 & 3 \\
4 & 5 & 6 \\
\hline
\end{tabular}
\end{verbatim}  


\subsection{Tables with Caption}
\begin{table}[h!]
\centering
\begin{tabular}{|c|c|c|}
\hline
X & Y & Z \\
\hline
7 & 8 & 9 \\
10 & 11 & 12 \\
\hline
\end{tabular}
\caption{A Simple Table}
\end{table}

\begin{verbatim}
\begin{table}[h!]
\centering
\begin{tabular}{|c|c|c|}
\hline
X & Y & Z \\
\hline
7 & 8 & 9 \\
10 & 11 & 12 \\
\hline
\end{tabular}
\caption{A Simple Table}
\end{table}
\end{verbatim}


\section{Listings}

\subsection{Itemize Environment}
\begin{itemize}
    \item First item
    \item Second item
    \item Third item
\end{itemize}

\begin{verbatim}
\begin{itemize}
    \item First item
    \item Second item
    \item Third item
\end{itemize}
\end{verbatim}

\subsection{Enumerate Environment}
\begin{enumerate}
    \item First item
    \item Second item
    \item Third item
\end{enumerate} 

\begin{verbatim}
\begin{enumerate}
    \item First item
    \item Second item
    \item Third item
\end{enumerate}
\end{verbatim}


\subsection{Text in boxes}

\begin{center}
\framebox{
    \begin{minipage}{0.8\textwidth}
    This is a text box in \LaTeX. You can use it to highlight important information or to create a visual separation from the surrounding text.
    \end{minipage}
}
\end{center}


\begin{verbatim}
\begin{center}
\framebox{
    \begin{minipage}{0.8\textwidth}
    This is a text box in \LaTeX. You can use it to highlight 
    important information or to create a visual 
    separation from the surrounding text.
    \end{minipage}
}
\end{center}
\end{verbatim}


\end{document}