\documentclass{article}

\usepackage{amsmath}
\usepackage{amsfonts}
\usepackage{amssymb}
\usepackage{graphicx}
\usepackage{hyperref}
\usepackage{float}
\usepackage[table]{xcolor}
\usepackage{tikzsymbols}
\usepackage{xcolor}
\usepackage{minted}
\usepackage{natbib}

\usepackage[
   a4paper,
   left=2.5cm,
   right=2.5cm,
   top=2.5cm,
   bottom=2.5cm
]{geometry}





\title{Spreadsheet (Excel, LibreOffice, etc.) Functions\footnote{
   This document is prepared for educational purposes only in the 
   information technology course series at Istanbul University, Faculty of Economics.
   The content includes formulas and functions commonly used in spreadsheet software such 
   as Microsoft Excel, LibreOffice, and Google Sheets, etc. The notes are for those 
   who already participated in the related courses. The source code of this content 
   is available at \url{https://github.com/jbytecode/sampledbs} in \LaTeX \; format. 
   \cite{MicrosoftExcelSupport} and \cite{LibreOfficeDocumentation} are used as references.
}}
\author{Mehmet Hakan Satman (PH.D.)\\
  Istanbul University, Faculty of Economics, Department of Econometrics\\}

\begin{document}

\maketitle

\tableofcontents

\newcommand{\CE}[1]{\cellcolor{blue!15}#1}


\section{Rounding Functions}

\subsection{ROUND}
\begin{table}[H]
\centering
\begin{tabular}{|c|c|c|c|c|c|c|}
\hline
   \CE{ } & \CE{A} & \CE{B} & \CE{C} & \CE{D} & \CE{E} & \CE{F}  \\
\hline
 \CE{1} &  & =ROUND(1.1234; 3)  &  &  &  &  \\
\hline
 \CE{2} &  &  &  &  &  &  \\
\end{tabular}
\end{table}

\begin{verbatim}
   1.123
\end{verbatim}

\begin{table}[H]
\centering
\begin{tabular}{|c|c|c|c|c|c|c|}
\hline
   \CE{ } & \CE{A} & \CE{B} & \CE{C} & \CE{D} & \CE{E} & \CE{F}  \\
\hline
 \CE{1} &  & =ROUND(1.1237; 3)  &  &  &  &  \\
\hline
 \CE{2} &  &  &  &  &  &  \\
\end{tabular}
\end{table}

\begin{verbatim}
   1.124
\end{verbatim}


\subsection{ROUNDUP}
\begin{table}[H]
\centering
\begin{tabular}{|c|c|c|c|c|c|c|}
\hline
   \CE{ } & \CE{A} & \CE{B} & \CE{C} & \CE{D} & \CE{E} & \CE{F}  \\
\hline
 \CE{1} &  & =ROUNDUP(1.1234; 3)  &  &  &  &  \\
\hline
 \CE{2} &  &  &  &  &  &  \\
\end{tabular}
\end{table}

\begin{verbatim}
   1.124
\end{verbatim}



\begin{table}[H]
\centering
\begin{tabular}{|c|c|c|c|c|c|c|}
\hline
   \CE{ } & \CE{A} & \CE{B} & \CE{C} & \CE{D} & \CE{E} & \CE{F}  \\
\hline
 \CE{1} &  & =ROUNDUP(1.1237; 3)  &  &  &  &  \\
\hline
 \CE{2} &  &  &  &  &  &  \\
\end{tabular}
\end{table}

\begin{verbatim}
   1.124
\end{verbatim}





\subsection{ROUNDDOWN}
\begin{table}[H]
\centering
\begin{tabular}{|c|c|c|c|c|c|c|}
\hline
   \CE{ } & \CE{A} & \CE{B} & \CE{C} & \CE{D} & \CE{E} & \CE{F}  \\
\hline
 \CE{1} &  & =ROUNDDOWN(1.1234; 3)  &  &  &  &  \\
\hline
 \CE{2} &  &  &  &  &  &  \\
\end{tabular}
\end{table}

\begin{verbatim}
   1.123
\end{verbatim}


\begin{table}[H]
\centering
\begin{tabular}{|c|c|c|c|c|c|c|}
\hline
   \CE{ } & \CE{A} & \CE{B} & \CE{C} & \CE{D} & \CE{E} & \CE{F}  \\
\hline
 \CE{1} &  & =ROUNDDOWN(1.1237; 3)  &  &  &  &  \\
\hline
 \CE{2} &  &  &  &  &  &  \\
\end{tabular}
\end{table}

\begin{verbatim}
   1.123
\end{verbatim}




\subsection{INT}
\begin{table}[H]
\centering
\begin{tabular}{|c|c|c|c|c|c|c|}
\hline
   \CE{ } & \CE{A} & \CE{B} & \CE{C} & \CE{D} & \CE{E} & \CE{F}  \\
\hline
 \CE{1} &  & =INT(1.1234)  &  &  &  &  \\
\hline
 \CE{2} &  & =INT(100.99)  &  &  &  &  \\
\end{tabular}
\end{table}

\begin{verbatim}
   1
   100
\end{verbatim}


\section{Mathematical Functions}

\subsection{SIGN}

\begin{table}[H]
\centering
\begin{tabular}{|c|c|c|c|c|c|c|}
\hline
   \CE{ } & \CE{A} & \CE{B} & \CE{C} & \CE{D} & \CE{E} & \CE{F}  \\
\hline
 \CE{1} &  & =SIGN(16)  &  &  &  &  \\
\hline
 \CE{2} &  &  &  &  &  &  \\
\end{tabular}
\end{table}

\begin{verbatim}
   1
\end{verbatim}

\begin{table}[H]
\centering
\begin{tabular}{|c|c|c|c|c|c|c|}
\hline
   \CE{ } & \CE{A} & \CE{B} & \CE{C} & \CE{D} & \CE{E} & \CE{F}  \\
\hline
 \CE{1} &  & =SIGN(-8)  &  &  &  &  \\
\hline
 \CE{2} &  &  &  &  &  &  \\
\end{tabular}
\end{table}

\begin{verbatim}
   -1
\end{verbatim}


\begin{table}[H]
\centering
\begin{tabular}{|c|c|c|c|c|c|c|}
\hline
   \CE{ } & \CE{A} & \CE{B} & \CE{C} & \CE{D} & \CE{E} & \CE{F}  \\
\hline
 \CE{1} &  & =SIGN(0)  &  &  &  &  \\
\hline
 \CE{2} &  &  &  &  &  &  \\
\end{tabular}
\end{table}

\begin{verbatim}
   0
\end{verbatim}


\subsection{ABS}

\begin{table}[H]
\centering
\begin{tabular}{|c|c|c|c|c|c|c|}
\hline
   \CE{ } & \CE{A} & \CE{B} & \CE{C} & \CE{D} & \CE{E} & \CE{F}  \\
\hline
 \CE{1} &  & =ABS(-25)  &  &  &  &  \\
\hline
 \CE{2} &  &  &  &  &  &  \\
\end{tabular}
\end{table}

\begin{verbatim}
   25
\end{verbatim}



\begin{table}[H]
\centering
\begin{tabular}{|c|c|c|c|c|c|c|}
\hline
   \CE{ } & \CE{A} & \CE{B} & \CE{C} & \CE{D} & \CE{E} & \CE{F}  \\
\hline
 \CE{1} &  & =ABS(25)  &  &  &  &  \\
\hline
 \CE{2} &  &  &  &  &  &  \\
\end{tabular}
\end{table}

\begin{verbatim}
   25
\end{verbatim}



\subsection{POWER}


\begin{table}[H]
\centering
\begin{tabular}{|c|c|c|c|c|c|c|}
\hline
   \CE{ } & \CE{A} & \CE{B} & \CE{C} & \CE{D} & \CE{E} & \CE{F}  \\
\hline
 \CE{1} &  & =POWER(2; 3)  &  &  &  &  \\
\hline
 \CE{2} &  &  &  &  &  &  \\
\end{tabular}
\end{table}

\begin{verbatim}
   8
\end{verbatim}


\begin{table}[H]
\centering
\begin{tabular}{|c|c|c|c|c|c|c|}
\hline
   \CE{ } & \CE{A} & \CE{B} & \CE{C} & \CE{D} & \CE{E} & \CE{F}  \\
\hline
 \CE{1} &  & =POWER(10; 4)  &  &  &  &  \\
\hline
 \CE{2} &  &  &  &  &  &  \\
\end{tabular}
\end{table}

\begin{verbatim}
   10000
\end{verbatim}



\subsection{LOG}

\begin{table}[H]
\centering
\begin{tabular}{|c|c|c|c|c|c|c|}
\hline
   \CE{ } & \CE{A} & \CE{B} & \CE{C} & \CE{D} & \CE{E} & \CE{F}  \\
\hline
 \CE{1} &  & =LOG(1000)  &  &  &  &  \\
\hline
 \CE{2} &  &  &  &  &  &  \\
\end{tabular}
\end{table}

\begin{verbatim}
   3
\end{verbatim}


\begin{table}[H]
\centering
\begin{tabular}{|c|c|c|c|c|c|c|}
\hline
   \CE{ } & \CE{A} & \CE{B} & \CE{C} & \CE{D} & \CE{E} & \CE{F}  \\
\hline
 \CE{1} &  & =LOG(8; 2)  &  &  &  &  \\
\hline
 \CE{2} &  &  &  &  &  &  \\
\end{tabular}
\end{table}

\begin{verbatim}
   3
\end{verbatim}


\subsection{LN}

\begin{table}[H]
\centering
\begin{tabular}{|c|c|c|c|c|c|c|}
\hline
   \CE{ } & \CE{A} & \CE{B} & \CE{C} & \CE{D} & \CE{E} & \CE{F}  \\
\hline
 \CE{1} &  & =LN(2.71828)  &  &  &  &  \\
\hline
 \CE{2} &  &  &  &  &  &  \\
\end{tabular}
\end{table}

\begin{verbatim}
   0.999999327347282
\end{verbatim}



\begin{table}[H]
\centering
\begin{tabular}{|c|c|c|c|c|c|c|}
\hline
   \CE{ } & \CE{A} & \CE{B} & \CE{C} & \CE{D} & \CE{E} & \CE{F}  \\
\hline
 \CE{1} &  & =LN(EXP(1))  &  &  &  &  \\
\hline
 \CE{2} &  &  &  &  &  &  \\
\end{tabular}
\end{table}

\begin{verbatim}
   1
\end{verbatim}



\subsection{LOG10}

\begin{table}[H]
\centering
\begin{tabular}{|c|c|c|c|c|c|c|}
\hline
   \CE{ } & \CE{A} & \CE{B} & \CE{C} & \CE{D} & \CE{E} & \CE{F}  \\
\hline
 \CE{1} &  & =LOG10(1000))  &  &  &  &  \\
\hline
 \CE{2} &  &  &  &  &  &  \\
\end{tabular}
\end{table}

\begin{verbatim}
   3
\end{verbatim}



\subsection{EXP}

$$
EXP(x) = e^x 
$$


\begin{table}[H]
\centering
\begin{tabular}{|c|c|c|c|c|c|c|}
\hline
   \CE{ } & \CE{A} & \CE{B} & \CE{C} & \CE{D} & \CE{E} & \CE{F}  \\
\hline
 \CE{1} &  & =EXP(1)  &  &  &  &  \\
\hline
 \CE{2} &  &  &  &  &  &  \\
\end{tabular}
\end{table}

\begin{verbatim}
   2.71828182845905
\end{verbatim}


\begin{table}[H]
\centering
\begin{tabular}{|c|c|c|c|c|c|c|}
\hline
   \CE{ } & \CE{A} & \CE{B} & \CE{C} & \CE{D} & \CE{E} & \CE{F}  \\
\hline
 \CE{1} &  & =LN(EXP(2))  &  &  &  &  \\
\hline
 \CE{2} &  &  &  &  &  &  \\
\end{tabular}
\end{table}

\begin{verbatim}
   2
\end{verbatim}


\subsection{SQRT}

\begin{table}[H]
\centering
\begin{tabular}{|c|c|c|c|c|c|c|}
\hline
   \CE{ } & \CE{A} & \CE{B} & \CE{C} & \CE{D} & \CE{E} & \CE{F}  \\
\hline
 \CE{1} &  & =SQRT(16)  &  &  &  &  \\
\hline
 \CE{2} &  &  &  &  &  &  \\
\end{tabular}
\end{table}

\begin{verbatim}
   4
\end{verbatim}




\subsection{PI}


\begin{table}[H]
\centering
\begin{tabular}{|c|c|c|c|c|c|c|}
\hline
   \CE{ } & \CE{A} & \CE{B} & \CE{C} & \CE{D} & \CE{E} & \CE{F}  \\
\hline
 \CE{1} &  & =PI()  &  &  &  &  \\
\hline
 \CE{2} &  &  &  &  &  &  \\
\end{tabular}
\end{table}

\begin{verbatim}
  3.14159265358979
\end{verbatim}




\subsection{SIN}


\begin{table}[H]
\centering
\begin{tabular}{|c|c|c|c|c|c|c|}
\hline
   \CE{ } & \CE{A} & \CE{B} & \CE{C} & \CE{D} & \CE{E} & \CE{F}  \\
\hline
 \CE{1} &  & =SIN(PI()/2)  &  &  &  &  \\
\hline
 \CE{2} &  &  &  &  &  &  \\
\end{tabular}
\end{table}

\begin{verbatim}
  1
\end{verbatim}




\subsection{COS}


\begin{table}[H]
\centering
\begin{tabular}{|c|c|c|c|c|c|c|}
\hline
   \CE{ } & \CE{A} & \CE{B} & \CE{C} & \CE{D} & \CE{E} & \CE{F}  \\
\hline
 \CE{1} &  & =COS(2 * PI())  &  &  &  &  \\
\hline
 \CE{2} &  &  &  &  &  &  \\
\end{tabular}
\end{table}

\begin{verbatim}
  1
\end{verbatim}



\subsection{RADIANS}

\begin{table}[H]
\centering
\begin{tabular}{|c|c|c|c|c|c|c|}
\hline
   \CE{ } & \CE{A} & \CE{B} & \CE{C} & \CE{D} & \CE{E} & \CE{F}  \\
\hline
 \CE{1} &  & =RADIANS(90)  &  &  &  &  \\
\hline
 \CE{2} &  &  &  &  &  &  \\
\end{tabular}
\end{table}

\begin{verbatim}
  1.5707963267949
\end{verbatim}




\subsection{DEGREES}

\begin{table}[H]
\centering
\begin{tabular}{|c|c|c|c|c|c|c|}
\hline
   \CE{ } & \CE{A} & \CE{B} & \CE{C} & \CE{D} & \CE{E} & \CE{F}  \\
\hline
 \CE{1} &  & =DEGREES(PI()/2)  &  &  &  &  \\
\hline
 \CE{2} &  &  &  &  &  &  \\
\end{tabular}
\end{table}

\begin{verbatim}
  90
\end{verbatim}






\subsection{ISNUMBER}


\begin{table}[H]
\centering
\begin{tabular}{|c|c|c|c|c|c|c|}
\hline
   \CE{ } & \CE{A} & \CE{B} & \CE{C} & \CE{D} & \CE{E} & \CE{F}  \\
\hline
 \CE{1} & 8 & =ISNUMBER(A1)  &  &  &  &  \\
\hline
 \CE{2} &  &  &  &  &  &  \\
\end{tabular}
\end{table}

\begin{verbatim}
  TRUE
\end{verbatim}





\begin{table}[H]
\centering
\begin{tabular}{|c|c|c|c|c|c|c|}
\hline
   \CE{ } & \CE{A} & \CE{B} & \CE{C} & \CE{D} & \CE{E} & \CE{F}  \\
\hline
 \CE{1} & Cat & =ISNUMBER(A1)  &  &  &  &  \\
\hline
 \CE{2} &  &  &  &  &  &  \\
\end{tabular}
\end{table}

\begin{verbatim}
  FALSE
\end{verbatim}





\subsection{ISEVEN}



\begin{table}[H]
\centering
\begin{tabular}{|c|c|c|c|c|c|c|}
\hline
   \CE{ } & \CE{A} & \CE{B} & \CE{C} & \CE{D} & \CE{E} & \CE{F}  \\
\hline
 \CE{1} & 8 & =ISEVEN(A1)  &  &  &  &  \\
\hline
 \CE{2} &  &  &  &  &  &  \\
\end{tabular}
\end{table}

\begin{verbatim}
  TRUE
\end{verbatim}


\begin{table}[H]
\centering
\begin{tabular}{|c|c|c|c|c|c|c|}
\hline
   \CE{ } & \CE{A} & \CE{B} & \CE{C} & \CE{D} & \CE{E} & \CE{F}  \\
\hline
 \CE{1} & 9 & =ISEVEN(A1)  &  &  &  &  \\
\hline
 \CE{2} &  &  &  &  &  &  \\
\end{tabular}
\end{table}

\begin{verbatim}
  FALSE
\end{verbatim}






\subsection{ISODD}



\begin{table}[H]
\centering
\begin{tabular}{|c|c|c|c|c|c|c|}
\hline
   \CE{ } & \CE{A} & \CE{B} & \CE{C} & \CE{D} & \CE{E} & \CE{F}  \\
\hline
 \CE{1} & 8 & =ISODD(A1)  &  &  &  &  \\
\hline
 \CE{2} &  &  &  &  &  &  \\
\end{tabular}
\end{table}

\begin{verbatim}
  FALSE
\end{verbatim}


\begin{table}[H]
\centering
\begin{tabular}{|c|c|c|c|c|c|c|}
\hline
   \CE{ } & \CE{A} & \CE{B} & \CE{C} & \CE{D} & \CE{E} & \CE{F}  \\
\hline
 \CE{1} & 9 & =ISODD(A1)  &  &  &  &  \\
\hline
 \CE{2} &  &  &  &  &  &  \\
\end{tabular}
\end{table}

\begin{verbatim}
  TRUE
\end{verbatim}





\subsection{ISBLANK}

\begin{table}[H]
\centering
\begin{tabular}{|c|c|c|c|c|c|c|}
\hline
   \CE{ } & \CE{A} & \CE{B} & \CE{C} & \CE{D} & \CE{E} & \CE{F}  \\
\hline
 \CE{1} &   & =ISBLANK(A1)  &  &  &  &  \\
\hline
 \CE{2} &  &  &  &  &  &  \\
\end{tabular}
\end{table}

\begin{verbatim}
  TRUE
\end{verbatim}


\begin{table}[H]
\centering
\begin{tabular}{|c|c|c|c|c|c|c|}
\hline
   \CE{ } & \CE{A} & \CE{B} & \CE{C} & \CE{D} & \CE{E} & \CE{F}  \\
\hline
 \CE{1} & Cats  & =ISBLANK(A1)  &  &  &  &  \\
\hline
 \CE{2} &  &  &  &  &  &  \\
\end{tabular}
\end{table}

\begin{verbatim}
  FALSE
\end{verbatim}



\subsection{ISERROR}

\begin{table}[H]
\centering
\begin{tabular}{|c|c|c|c|c|c|c|}
\hline
   \CE{ } & \CE{A} & \CE{B} & \CE{C} & \CE{D} & \CE{E} & \CE{F}  \\
\hline
 \CE{1} &   & = 9/0  &  &  &  &  \\
\hline
 \CE{2} &   & = ISERROR(A1)  &  &  &  &  \\
\end{tabular}
\end{table}


\begin{verbatim}
  A2 = TRUE
\end{verbatim}



\begin{table}[H]
\centering
\begin{tabular}{|c|c|c|c|c|c|c|}
\hline
   \CE{ } & \CE{A} & \CE{B} & \CE{C} & \CE{D} & \CE{E} & \CE{F}  \\
\hline
 \CE{1} &   & \#DIV/0!  &  &  &  &  \\
\hline
 \CE{2} &   & = ISERROR(A1)  &  &  &  &  \\
\end{tabular}
\end{table}


\begin{verbatim}
  TRUE
\end{verbatim}


\subsection{BASE}
\begin{table}[H]
\centering
\begin{tabular}{|c|c|c|c|c|c|c|}
\hline
   \CE{ } & \CE{A} & \CE{B} & \CE{C} & \CE{D} & \CE{E} & \CE{F}  \\
\hline
 \CE{1} &   & =BASE(16; 2; 8)  &  &  &  &  \\
\hline
 \CE{2} &   & =BASE(255, 2; 8)  &  &  &  &  \\
\hline
 \CE{3} &   & =BASE(3; 2; 8)  &  &  &  &  \\
\end{tabular}
\end{table}

\begin{verbatim}
  00010000
  11111111
  00000011
\end{verbatim}



\subsection{MOD}
\begin{table}[H]
\centering
\begin{tabular}{|c|c|c|c|c|c|c|}
\hline
   \CE{ } & \CE{A} & \CE{B} & \CE{C} & \CE{D} & \CE{E} & \CE{F}  \\
\hline
 \CE{1} &   & =MOD(5; 2)  &  &  &  &  \\
\hline
 \CE{2} &   & =MOD(10; 9)  &  &  &  &  \\
\hline
 \CE{3} &   & =MOD(8; 3)  &  &  &  &  \\
\end{tabular}
\end{table}

\begin{verbatim}
  1
  1
  2
\end{verbatim}



\subsection{ROMAN}
\begin{table}[H]
\centering
\begin{tabular}{|c|c|c|c|c|c|c|}
\hline
   \CE{ } & \CE{A} & \CE{B} & \CE{C} & \CE{D} & \CE{E} & \CE{F}  \\
\hline
 \CE{1} &   & =ROMAN(1453)  &  &  &  &  \\
\hline
 \CE{2} &   & =ROMAN(2025)  &  &  &  &  \\
\hline
 \CE{3} &   & =ROMAN(8)  &  &  &  &  \\
\hline
\CE{4} &   & =ROMAN(105)  &  &  &  &  \\
\hline 
\CE{5} &   & =ROMAN(50)  &  &  &  &  \\
\end{tabular}
\end{table}

\begin{verbatim}
MCDLIII
MMXXV
VIII
CV
L
\end{verbatim}


\subsection{Function Composition}


\subsubsection{Example 1}
\begin{verbatim}
=SQRT(POWER(3;2) + POWER(4;2))
\end{verbatim}

Answer:

\begin{verbatim}
  5
\end{verbatim}

\subsubsection{Example 2}
\begin{verbatim}
=LOG10(POWER(10;4))
\end{verbatim}

Answer:

\begin{verbatim}
  4
\end{verbatim}

\subsubsection{Example 3}
\begin{verbatim}
=ROUNDUP(SQRT(50);3)
\end{verbatim}

Answer:
\begin{verbatim}
  7.072
\end{verbatim}

\subsubsection{Example 4}
\begin{verbatim}
=ABS(MOD(29;6))
\end{verbatim}

Answer:
\begin{verbatim}
  5
\end{verbatim}


\section{Statistical Functions}

\subsection{MIN}

\begin{table}[H]
\centering
\begin{tabular}{|c|c|c|c|c|c|c|}
\hline
   \CE{ } & \CE{A} & \CE{B} & \CE{C} & \CE{D} & \CE{E} & \CE{F}  \\
\hline
 \CE{1} & 5  &   &  &  &  &  \\
\hline
 \CE{2} & 7 &  &  &  &  &  \\ 
\hline
 \CE{3} & 12 &  &  &  &  &  \\  
\hline
 \CE{4} & 11 &  &  &  &  &  \\ 
\hline
 \CE{5} & =MIN(A1:A4) &  &  &  &  &  \\ 
\end{tabular}
\end{table}

\begin{verbatim}
   5
\end{verbatim}




\subsection{MAX}

\begin{table}[H]
\centering
\begin{tabular}{|c|c|c|c|c|c|c|}
\hline
   \CE{ } & \CE{A} & \CE{B} & \CE{C} & \CE{D} & \CE{E} & \CE{F}  \\
\hline
 \CE{1} & 5  &   &  &  &  &  \\
\hline
 \CE{2} & 7 &  &  &  &  &  \\ 
\hline
 \CE{3} & 12 &  &  &  &  &  \\  
\hline
 \CE{4} & 11 &  &  &  &  &  \\ 
\hline
 \CE{5} & =MAX(A1:A4) &  &  &  &  &  \\ 
\end{tabular}
\end{table}

\begin{verbatim}
   12
\end{verbatim}






\subsection{VAR}

\begin{table}[H]
\centering
\begin{tabular}{|c|c|c|c|c|c|c|}
\hline
   \CE{ } & \CE{A} & \CE{B} & \CE{C} & \CE{D} & \CE{E} & \CE{F}  \\
\hline
 \CE{1} & 5  &   &  &  &  &  \\
\hline
 \CE{2} & 7 &  &  &  &  &  \\ 
\hline
 \CE{3} & 12 &  &  &  &  &  \\  
\hline
 \CE{4} & 11 &  &  &  &  &  \\ 
\hline
 \CE{5} & =VAR(A1:A4) &  &  &  &  &  \\ 
\end{tabular}
\end{table}

\begin{verbatim}
   10.9166666666667
\end{verbatim}



\subsection{STDEV}

\begin{table}[H]
\centering
\begin{tabular}{|c|c|c|c|c|c|c|}
\hline
   \CE{ } & \CE{A} & \CE{B} & \CE{C} & \CE{D} & \CE{E} & \CE{F}  \\
\hline
 \CE{1} & 5  &   &  &  &  &  \\
\hline
 \CE{2} & 7 &  &  &  &  &  \\ 
\hline
 \CE{3} & 12 &  &  &  &  &  \\  
\hline
 \CE{4} & 11 &  &  &  &  &  \\ 
\hline
 \CE{5} & =STDEV(A1:A4) &  &  &  &  &  \\ 
\end{tabular}
\end{table}

\begin{verbatim}
   3.30403793359984
\end{verbatim}




\subsection{SUM}

\begin{table}[H]
\centering
\begin{tabular}{|c|c|c|c|c|c|c|}
\hline
   \CE{ } & \CE{A} & \CE{B} & \CE{C} & \CE{D} & \CE{E} & \CE{F}  \\
\hline
 \CE{1} & 5  &   &  &  &  &  \\
\hline
 \CE{2} & 7 &  &  &  &  &  \\ 
\hline
 \CE{3} & 12 &  &  &  &  &  \\  
\hline
 \CE{4} & 11 &  &  &  &  &  \\ 
\hline
 \CE{5} & =SUM(A1:A4) &  &  &  &  &  \\ 
\end{tabular}
\end{table}

\begin{verbatim}
   35
\end{verbatim}








\subsection{PRODUCT}

\begin{table}[H]
\centering
\begin{tabular}{|c|c|c|c|c|c|c|}
\hline
   \CE{ } & \CE{A} & \CE{B} & \CE{C} & \CE{D} & \CE{E} & \CE{F}  \\
\hline
 \CE{1} & 5  &   &  &  &  &  \\
\hline
 \CE{2} & 7 &  &  &  &  &  \\ 
\hline
 \CE{3} & 12 &  &  &  &  &  \\  
\hline
 \CE{4} & 11 &  &  &  &  &  \\ 
\hline
 \CE{5} & =PRODUCT(A1:A4) &  &  &  &  &  \\ 
\end{tabular}
\end{table}

\begin{verbatim}
   4620
\end{verbatim}





\subsection{SUMPRODUCT}

\begin{table}[H]
\centering
\begin{tabular}{|c|c|c|c|c|c|c|}
\hline
   \CE{ } & \CE{A} & \CE{B} & \CE{C} & \CE{D} & \CE{E} & \CE{F}  \\
\hline
 \CE{1} & 5  &  4 &  &  &  &  \\
\hline
 \CE{2} & 7 &  0 &  &  &  &  \\ 
\hline
 \CE{3} & 12 & 2 &  &  &  &  \\  
\hline
 \CE{4} & 11 &  3 &  &  &  &  \\ 
\hline
 \CE{5} & =SUMPRODUCT(A1:A4; B1:B4) &  &  &  &  &  \\ 
\end{tabular}
\end{table}

\begin{verbatim}
   77
\end{verbatim}





\subsection{AVERAGE}

\begin{table}[H]
\centering
\begin{tabular}{|c|c|c|c|c|c|c|}
\hline
   \CE{ } & \CE{A} & \CE{B} & \CE{C} & \CE{D} & \CE{E} & \CE{F}  \\
\hline
 \CE{1} & 5  &   &  &  &  &  \\
\hline
 \CE{2} & 7 &  &  &  &  &  \\ 
\hline
 \CE{3} & 12 &  &  &  &  &  \\  
\hline
 \CE{4} & 11 &  &  &  &  &  \\ 
\hline
 \CE{5} & =AVERAGE(A1:A4) &  &  &  &  &  \\ 
\end{tabular}
\end{table}

\begin{verbatim}
   8.75
\end{verbatim}






\subsection{COUNT}

\begin{table}[H]
\centering
\begin{tabular}{|c|c|c|c|c|c|c|}
\hline
   \CE{ } & \CE{A} & \CE{B} & \CE{C} & \CE{D} & \CE{E} & \CE{F}  \\
\hline
 \CE{1} & 5  &   &  &  &  &  \\
\hline
 \CE{2} & 7 &  &  &  &  &  \\ 
\hline
 \CE{3} & 12 &  &  &  &  &  \\  
\hline
 \CE{4} & 11 &  &  &  &  &  \\ 
\hline
 \CE{5} & =COUNT(A1:A4) &  &  &  &  &  \\ 
\end{tabular}
\end{table}

\begin{verbatim}
   4
\end{verbatim}







\subsection{MEDIAN}

\begin{table}[H]
\centering
\begin{tabular}{|c|c|c|c|c|c|c|}
\hline
   \CE{ } & \CE{A} & \CE{B} & \CE{C} & \CE{D} & \CE{E} & \CE{F}  \\
\hline
 \CE{1} & 5  &   &  &  &  &  \\
\hline
 \CE{2} & 7 &  &  &  &  &  \\ 
\hline
 \CE{3} & 12 &  &  &  &  &  \\  
\hline
 \CE{4} & 11 &  &  &  &  &  \\ 
\hline
 \CE{5} & =MEDIAN(A1:A4) &  &  &  &  &  \\ 
\end{tabular}
\end{table}

\begin{verbatim}
   9.0
\end{verbatim}


\subsection{PERCENTILE}

\begin{table}[H]
\centering
\begin{tabular}{|c|c|c|c|c|c|c|}
\hline
   \CE{ } & \CE{A} & \CE{B} & \CE{C} & \CE{D} & \CE{E} & \CE{F}  \\
\hline
 \CE{1} & 5  &   &  &  &  &  \\
\hline
 \CE{2} & 7 &  &  &  &  &  \\ 
\hline
 \CE{3} & 12 &  &  &  &  &  \\  
\hline
 \CE{4} & 11 &  &  &  &  &  \\ 
\hline
 \CE{5} & =PERCENTILE(A1:A4; 0) &  &  &  &  &  \\ 
\end{tabular}
\end{table}


\begin{verbatim}
   5
\end{verbatim}




\begin{table}[H]
\centering
\begin{tabular}{|c|c|c|c|c|c|c|}
\hline
   \CE{ } & \CE{A} & \CE{B} & \CE{C} & \CE{D} & \CE{E} & \CE{F}  \\
\hline
 \CE{1} & 5  &   &  &  &  &  \\
\hline
 \CE{2} & 7 &  &  &  &  &  \\ 
\hline
 \CE{3} & 12 &  &  &  &  &  \\  
\hline
 \CE{4} & 11 &  &  &  &  &  \\ 
\hline
 \CE{5} & =PERCENTILE(A1:A4; 1) &  &  &  &  &  \\ 
\end{tabular}
\end{table}


\begin{verbatim}
   12
\end{verbatim}





\begin{table}[H]
\centering
\begin{tabular}{|c|c|c|c|c|c|c|}
\hline
   \CE{ } & \CE{A} & \CE{B} & \CE{C} & \CE{D} & \CE{E} & \CE{F}  \\
\hline
 \CE{1} & 5  &   &  &  &  &  \\
\hline
 \CE{2} & 7 &  &  &  &  &  \\ 
\hline
 \CE{3} & 12 &  &  &  &  &  \\  
\hline
 \CE{4} & 11 &  &  &  &  &  \\ 
\hline
 \CE{5} & =PERCENTILE(A1:A4; 0.5) &  &  &  &  &  \\ 
\end{tabular}
\end{table}


\begin{verbatim}
   9
\end{verbatim}



\begin{table}[H]
\centering
\begin{tabular}{|c|c|c|c|c|c|c|}
\hline
   \CE{ } & \CE{A} & \CE{B} & \CE{C} & \CE{D} & \CE{E} & \CE{F}  \\
\hline
 \CE{1} & 5  &   &  &  &  &  \\
\hline
 \CE{2} & 7 &  &  &  &  &  \\ 
\hline
 \CE{3} & 12 &  &  &  &  &  \\  
\hline
 \CE{4} & 11 &  &  &  &  &  \\ 
\hline
 \CE{5} & =PERCENTILE(A1:A4; 0.25) &  &  &  &  &  \\ 
\end{tabular}
\end{table}


\begin{verbatim}
   6.5
\end{verbatim}





\begin{table}[H]
\centering
\begin{tabular}{|c|c|c|c|c|c|c|}
\hline
   \CE{ } & \CE{A} & \CE{B} & \CE{C} & \CE{D} & \CE{E} & \CE{F}  \\
\hline
 \CE{1} & 5  &   &  &  &  &  \\
\hline
 \CE{2} & 7 &  &  &  &  &  \\ 
\hline
 \CE{3} & 12 &  &  &  &  &  \\  
\hline
 \CE{4} & 11 &  &  &  &  &  \\ 
\hline
 \CE{5} & =PERCENTILE(A1:A4; 0.75) &  &  &  &  &  \\ 
\end{tabular}
\end{table}


\begin{verbatim}
   11.25
\end{verbatim}



\subsection{QUARTILE}


\begin{table}[H]
\centering
\begin{tabular}{|c|c|c|c|c|c|c|}
\hline
   \CE{ } & \CE{A} & \CE{B} & \CE{C} & \CE{D} & \CE{E} & \CE{F}  \\
\hline
 \CE{1} & 5  &   &  &  &  &  \\
\hline
 \CE{2} & 7 &  &  &  &  &  \\ 
\hline
 \CE{3} & 12 &  &  &  &  &  \\  
\hline
 \CE{4} & 11 &  &  &  &  &  \\ 
\hline
 \CE{5} & =QUARTILE(A1:A4; 0) &  &  &  &  &  \\ 
\end{tabular}
\end{table}


\begin{verbatim}
   5
\end{verbatim}






\begin{table}[H]
\centering
\begin{tabular}{|c|c|c|c|c|c|c|}
\hline
   \CE{ } & \CE{A} & \CE{B} & \CE{C} & \CE{D} & \CE{E} & \CE{F}  \\
\hline
 \CE{1} & 5  &   &  &  &  &  \\
\hline
 \CE{2} & 7 &  &  &  &  &  \\ 
\hline
 \CE{3} & 12 &  &  &  &  &  \\  
\hline
 \CE{4} & 11 &  &  &  &  &  \\ 
\hline
 \CE{5} & =QUARTILE(A1:A4; 1) &  &  &  &  &  \\ 
\end{tabular}
\end{table}


\begin{verbatim}
   6.5
\end{verbatim}








\begin{table}[H]
\centering
\begin{tabular}{|c|c|c|c|c|c|c|}
\hline
   \CE{ } & \CE{A} & \CE{B} & \CE{C} & \CE{D} & \CE{E} & \CE{F}  \\
\hline
 \CE{1} & 5  &   &  &  &  &  \\
\hline
 \CE{2} & 7 &  &  &  &  &  \\ 
\hline
 \CE{3} & 12 &  &  &  &  &  \\  
\hline
 \CE{4} & 11 &  &  &  &  &  \\ 
\hline
 \CE{5} & =QUARTILE(A1:A4; 2) &  &  &  &  &  \\ 
\end{tabular}
\end{table}


\begin{verbatim}
   9
\end{verbatim}







\begin{table}[H]
\centering
\begin{tabular}{|c|c|c|c|c|c|c|}
\hline
   \CE{ } & \CE{A} & \CE{B} & \CE{C} & \CE{D} & \CE{E} & \CE{F}  \\
\hline
 \CE{1} & 5  &   &  &  &  &  \\
\hline
 \CE{2} & 7 &  &  &  &  &  \\ 
\hline
 \CE{3} & 12 &  &  &  &  &  \\  
\hline
 \CE{4} & 11 &  &  &  &  &  \\ 
\hline
 \CE{5} & =QUARTILE(A1:A4; 3) &  &  &  &  &  \\ 
\end{tabular}
\end{table}


\begin{verbatim}
   11.25
\end{verbatim}



\begin{table}[H]
\centering
\begin{tabular}{|c|c|c|c|c|c|c|}
\hline
   \CE{ } & \CE{A} & \CE{B} & \CE{C} & \CE{D} & \CE{E} & \CE{F}  \\
\hline
 \CE{1} & 5  &   &  &  &  &  \\
\hline
 \CE{2} & 7 &  &  &  &  &  \\ 
\hline
 \CE{3} & 12 &  &  &  &  &  \\  
\hline
 \CE{4} & 11 &  &  &  &  &  \\ 
\hline
 \CE{5} & =QUARTILE(A1:A4; 4) &  &  &  &  &  \\ 
\end{tabular}
\end{table}


\begin{verbatim}
   12
\end{verbatim}



\subsection{SKEW}


\begin{table}[H]
\centering
\begin{tabular}{|c|c|c|c|c|c|c|}
\hline
   \CE{ } & \CE{A} & \CE{B} & \CE{C} & \CE{D} & \CE{E} & \CE{F}  \\
\hline
 \CE{1} & 5  &   &  &  &  &  \\
\hline
 \CE{2} & 7 &  &  &  &  &  \\ 
\hline
 \CE{3} & 12 &  &  &  &  &  \\  
\hline
 \CE{4} & 11 &  &  &  &  &  \\ 
\hline
 \CE{5} & =SKEW(A1:A4) &  &  &  &  &  \\ 
\end{tabular}
\end{table}


\begin{verbatim}
   -0.592518588276328
\end{verbatim}







\begin{table}[H]
\centering
\begin{tabular}{|c|c|c|c|c|c|c|}
\hline
   \CE{ } & \CE{A} & \CE{B} & \CE{C} & \CE{D} & \CE{E} & \CE{F}  \\
\hline
 \CE{1} & 5  &   &  &  &  &  \\
\hline
 \CE{2} & 7 &  &  &  &  &  \\ 
\hline
 \CE{3} & 12 &  &  &  &  &  \\  
\hline
 \CE{4} & 11 &  &  &  &  &  \\ 
\hline
 \CE{5} & =KURT(A1:A4) &  &  &  &  &  \\ 
\end{tabular}
\end{table}


\begin{verbatim}
   -3.86900530272129
\end{verbatim}








\subsection{CORREL}

\begin{table}[H]
\centering
\begin{tabular}{|c|c|c|c|c|c|c|}
\hline
   \CE{ } & \CE{A} & \CE{B} & \CE{C} & \CE{D} & \CE{E} & \CE{F}  \\
\hline
 \CE{1} & 5  &  10 &  &  &  &  \\
\hline
 \CE{2} & 7 &  14 &  &  &  &  \\ 
\hline
 \CE{3} & 12 &  24&  &  &  &  \\  
\hline
 \CE{4} & 11 &  22&  &  &  &  \\ 
\hline
 \CE{5} & =CORREL(A1:A4; B1:B4) &  &  &  &  &  \\ 
\end{tabular}
\end{table}


\begin{verbatim}
   1
\end{verbatim}








\begin{table}[H]
\centering
\begin{tabular}{|c|c|c|c|c|c|c|}
\hline
   \CE{ } & \CE{A} & \CE{B} & \CE{C} & \CE{D} & \CE{E} & \CE{F}  \\
\hline
 \CE{1} & 5  &  22 &  &  &  &  \\
\hline
 \CE{2} & 7 &  24 &  &  &  &  \\ 
\hline
 \CE{3} & 12 &  14&  &  &  &  \\  
\hline
 \CE{4} & 11 &  10&  &  &  &  \\ 
\hline
 \CE{5} & =CORREL(A1:A4; B1:B4) &  &  &  &  &  \\ 
\end{tabular}
\end{table}


\begin{verbatim}
   -0.862595419847328
\end{verbatim}




\subsection{FACT}
\begin{table}[H]
\centering
\begin{tabular}{|c|c|c|c|c|c|c|}
\hline
   \CE{ } & \CE{A} & \CE{B} & \CE{C} & \CE{D} & \CE{E} & \CE{F}  \\
\hline
 \CE{1} & =FACT(5)  &   &  &  &  &  \\
\hline
 \CE{2} &  &   &  &  &  &  \\ 
\hline
\end{tabular}
\end{table}

\begin{verbatim}
   120
\end{verbatim}


\subsection{COMBIN}
\begin{table}[H]
\centering
\begin{tabular}{|c|c|c|c|c|c|c|}
\hline
   \CE{ } & \CE{A} & \CE{B} & \CE{C} & \CE{D} & \CE{E} & \CE{F}  \\
\hline
 \CE{1} & =COMBIN(5; 2)  &   &  &  &  &  \\
\hline
 \CE{2} &  &   &  &  &  &  \\ 
\hline
\end{tabular}
\end{table}

$$
\text{nCr} = \frac{n!}{r!(n-r)!} = \frac{5!}{2!(5-2)!} = \frac{5 \times 4 \times 3!}{2 \times 1 \times 3!} = \frac{20}{2} = 10
$$

\begin{verbatim}
   10
\end{verbatim}


\subsection{SUMSQ}
\begin{table}[H]
\centering
\begin{tabular}{|c|c|c|c|c|c|c|}
\hline
   \CE{ } & \CE{A} & \CE{B} & \CE{C} & \CE{D} & \CE{E} & \CE{F}  \\
\hline
 \CE{1} & 1  &   &  &  &  &  \\
\hline
 \CE{2} & 2 &   &  &  &  &  \\ 
\hline
 \CE{3} & 3 &   &  &  &  &  \\  
\hline
 \CE{4} & 4 &   &  &  &  &  \\ 
\hline
 \CE{5} & =SUMSQ(A1:A4) &  &  &  &  &  \\ 
\end{tabular}
\end{table}

\begin{verbatim}
   30
\end{verbatim}




\section{Logical Functions}

\subsection{AND}
\begin{table}[H]
\centering
\begin{tabular}{|c|c|c|c|c|c|c|}
\hline
   \CE{ } & \CE{A} & \CE{B} & \CE{C} & \CE{D} & \CE{E} & \CE{F}  \\
\hline
 \CE{1} &  & TRUE  &  FALSE &  &  &  \\
\hline
 \CE{2} &  & TRUE  &  TRUE &  &  &  \\
\hline
 \CE{3} &  & FALSE  &  FALSE &  &  &  \\
\hline
 \CE{4} &  & =AND(B1:C1) &  & &  &  \\
\hline
   \CE{5} &  & =AND(B2:C2) &  & &  &  \\
\hline
   \CE{6} &  & =AND(B3:C3) &  & &  &  \\
\end{tabular}
\end{table}

\begin{verbatim}
   FALSE 
   TRUE
   FALSE
\end{verbatim}




\subsection{OR}
\begin{table}[H]
\centering
\begin{tabular}{|c|c|c|c|c|c|c|}
\hline
   \CE{ } & \CE{A} & \CE{B} & \CE{C} & \CE{D} & \CE{E} & \CE{F}  \\
\hline
 \CE{1} &  & TRUE  &  FALSE &  &  &  \\
\hline
 \CE{2} &  & TRUE  &  TRUE &  &  &  \\
\hline
 \CE{3} &  & FALSE  &  FALSE &  &  &  \\
\hline
 \CE{4} &  & =OR(B1:C1) &  & &  &  \\
\hline
   \CE{5} &  & =OR(B2:C2) &  & &  &  \\
\hline
   \CE{6} &  & =OR(B3:C3) &  & &  &  \\
\end{tabular}
\end{table}

\begin{verbatim}
   TRUE   
   TRUE
   FALSE
\end{verbatim}



\subsection{NOT}
\begin{table}[H]
\centering
\begin{tabular}{|c|c|c|c|c|c|c|}
\hline
   \CE{ } & \CE{A} & \CE{B} & \CE{C} & \CE{D} & \CE{E} & \CE{F}  \\
\hline
 \CE{1} &  & TRUE  &   &  &  &  \\
\hline
 \CE{2} &  & TRUE  &   &  &  &  \\
\hline
 \CE{3} &  & FALSE  &   &  &  &  \\
\hline
 \CE{4} &  & =NOT(B1) &  & &  &  \\
\hline
   \CE{5} &  & =NOT(B2) &  & &  &  \\
\hline
   \CE{6} &  & =NOT(B3) &  & &  &  \\
\end{tabular}
\end{table}

\begin{verbatim}
   FALSE   
   FALSE
   TRUE
\end{verbatim}





\subsection{XOR}
\begin{table}[H]
\centering
\begin{tabular}{|c|c|c|c|c|c|c|}
\hline
   \CE{ } & \CE{A} & \CE{B} & \CE{C} & \CE{D} & \CE{E} & \CE{F}  \\
\hline
 \CE{1} &  & TRUE  &  FALSE &  &  &  \\
\hline
 \CE{2} &  & TRUE  &  TRUE &  &  &  \\
\hline
 \CE{3} &  & FALSE  &  TRUE &  &  &  \\
\hline
\CE{4} &   & FALSE & FALSE &  &  &  \\
\hline
 \CE{5} &  & =XOR(B1:C1) &  & &  &  \\
\hline
   \CE{6} &  & =XOR(B2:C2) &  & &  &  \\
\hline
   \CE{7} &  & =XOR(B3:C3) &  & &  &  \\
\end{tabular}
\end{table}

\begin{verbatim}
   TRUE   
   FALSE
   TRUE
   FALSE
\end{verbatim}



\subsection{IF}
\begin{table}[H]
\centering
\begin{tabular}{|c|c|c|c|c|c|c|}
\hline
   \CE{ } & \CE{A} & \CE{B} & \CE{C} & \CE{D} & \CE{E} & \CE{F}  \\
\hline
 \CE{1} &  5 &   6 &  12 &  -1 & 10 &  \\
\hline
 \CE{2} &  =IF(A1 = 5; TRUE; FALSE) &  &  &  &  &  \\
\hline
 \CE{3} &  =IF(AND(A1 $>=$ 5; SIGN(D1) = -1); "YES"; "NO") &  &   &  &  &  \\
\hline
\CE{4} &  =IF(AND(A1 = 5; B1 = 6; C1 $>$ 5); "YES"; "NO") &  &  &  &  &  \\
\hline
 \CE{5} & =IF(A1 $<$ 0; "NEG"; IF(A1 = 0; "ZERO"; "POS")) &  &  & &  &  \\
\hline
   \CE{6} &  &  &  & &  &  \\
\hline
   \CE{7} &  &  &  & &  &  \\
\end{tabular}
\end{table}

\begin{verbatim}
TRUE 
YES
YES
POS
\end{verbatim}




\subsubsection{IF...ELSEIF...ELSEIF...ELSE}
\begin{table}[H]
\centering
\begin{tabular}{|c|c|c|c|c|c|c|}
\hline
   \CE{ } & \CE{A} & \CE{B} & \CE{C} & \CE{D} & \CE{E} & \CE{F}  \\
\hline
 \CE{1} & 5  &  22 &  &  &  &  \\
\hline
 \CE{2} & 7 &  24 &  &  &  &  \\ 
\hline
\end{tabular}
\end{table}

Write an IF statement to handle this situation:
\begin{itemize}
\item If the value in A1 is less than 10, return "LOW"
\item If the value in A1 is between 10 and 20 (inclusive), return "MEDIUM"
\item If the value in A1 is between 21 and 30 (inclusive), return "HIGH"
\item If the value in A1 is greater than 30, return "VERY HIGH"
\end{itemize}

\begin{verbatim}
=IF(A1 < 10; "LOW"; 
   IF(AND(A1 >= 10; A1 <= 20); "MEDIUM"; 
      IF(AND(A1 >= 21; A1 <= 30); "HIGH"; 
         "VERY HIGH")))
\end{verbatim}


A shorter version:

\begin{verbatim}
=IF(A1 < 10; "LOW"; 
   IF(A1 <= 20; "MEDIUM"; 
      IF(A1 <= 30; "HIGH"; 
         "VERY HIGH")))
\end{verbatim}

The answer is 

\begin{verbatim}
   LOW
\end{verbatim}

\noindent Use \texttt{IFS} function (if your Excel supports it):

\begin{verbatim}
=IFS(A1 < 10; "LOW";
     AND(A1 >= 10; A1 <= 20); "MEDIUM";
     AND(A1 >= 21; A1 <= 30); "HIGH";
     A1 > 30; "VERY HIGH")
\end{verbatim}




\subsubsection{How many real roots?}
\begin{table}[H]
\centering
\begin{tabular}{|c|c|c|c|c|c|c|}
\hline
   \CE{ } & \CE{A} & \CE{B} & \CE{C} & \CE{D} & \CE{E} & \CE{F}  \\
\hline
 \CE{1} & -1  &  5 & 100  &  &  &  \\
\hline
 \CE{2} & =POWER(B1; 2) - 4 * A1 * C1  &    &  &  &  &  \\ 
\hline
\end{tabular}
\end{table}

Write an IF statement (indeed they are expressions in Excel) to determine how many real roots the 
quadratic equation $Ax^2 + Bx + C = 0$ has, based on the values in A1, B1, and C1. 
Return 0 (No real roots), 1 (repeated root), or 2 (two real roots).


\begin{verbatim}
=IF(B2 < 0; 0; 
   IF(B2 = 0; 1; 2))
\end{verbatim}


\noindent Use \texttt{IFS} function (if your Excel supports it):

\begin{verbatim}
=IFS(B2 < 0; 0;
     B2 = 0; 1;
     B2 > 0; 2)
\end{verbatim}



\subsubsection{How is the weather outside?}
Suppose that the temperature is in cell A1 (in degrees Celsius). 
Write an IF statement to return the following:
\begin{itemize}
\item If the temperature is less than or equal to 0, return "Freezing"
\item If the temperature is greater than 0 but less than or equal to 15, return "Cold"
\item If the temperature is greater than 15 but less than or equal to 25, return "Warm"
\item If the temperature is greater than 25 but less than or equal to 35, return "Hot"
\item If the temperature is greater than 35, return "Too Hot"
\end{itemize}

\begin{verbatim}
=IF(A1 <= 0; "Freezing";
   IF(A1 <= 15; "Cold";
      IF(A1 <= 25; "Warm";
         IF(A1 <= 35; "Hot";
            "Too Hot"))))
\end{verbatim}


\noindent Use \texttt{IFS} function (if your Excel supports it):

\begin{verbatim}
=IFS(A1 <= 0; "Freezing";
     A1 <= 15; "Cold";
     A1 <= 25; "Warm";
     A1 <= 35; "Hot";
     A1 > 35; "Too Hot")
\end{verbatim}




\subsection{IFERROR}

\begin{table}[H]
\centering
\begin{tabular}{|c|c|c|c|c|c|c|}
\hline
   \CE{ } & \CE{A} & \CE{B} & \CE{C} & \CE{D} & \CE{E} & \CE{F}  \\
\hline
 \CE{1} &  =IFERROR(9/0; "Undefined")   &    &  &  &  &  \\
\hline
 \CE{2} &     &    &  &  &  &  \\ 
\hline
\end{tabular}
\end{table}


\begin{verbatim}
   Undefined
\end{verbatim}


\section{Random Numbers}
\subsection{RAND}

\begin{table}[H]
\centering
\begin{tabular}{|c|c|c|c|c|c|c|}
\hline
   \CE{ } & \CE{A} & \CE{B} & \CE{C} & \CE{D} & \CE{E} & \CE{F}  \\
\hline
 \CE{1} &  & =RAND()  &  &  &  &  \\
\hline
 \CE{2} &  &  &  &  &  &  \\
\end{tabular}
\end{table}

\begin{verbatim}
   // A random number between 0 and 1, e.g.
   0.5432101234
\end{verbatim}



\subsection{RANDBETWEEN}

\begin{table}[H]
\centering
\begin{tabular}{|c|c|c|c|c|c|c|}
\hline
   \CE{ } & \CE{A} & \CE{B} & \CE{C} & \CE{D} & \CE{E} & \CE{F}  \\
\hline
 \CE{1} &  & =RANDBETWEEN(1; 10)  &  &  &  &  \\
\hline
 \CE{2} &  &  &  &  &  &  \\
\end{tabular}
\end{table}

\begin{verbatim}
   // A random integer between 1 and 10, e.g.
   // The range is inclusive, so 1 and 10 are possible outputs
   7
\end{verbatim}



\section{Date and Time Functions}

\subsection{NOW}
\begin{table}[H]
\centering
\begin{tabular}{|c|c|c|c|c|c|c|}
\hline
   \CE{ } & \CE{A} & \CE{B} & \CE{C} & \CE{D} & \CE{E} & \CE{F}  \\
\hline
 \CE{1} &  & =NOW()  &  &  &  &  \\
\hline
 \CE{2} &  &  &  &  &  &  \\
\end{tabular}
\end{table}

\begin{verbatim}
   01/12/25 08:10 PM
\end{verbatim}




\subsection{YEAR}
\begin{table}[H]
\centering
\begin{tabular}{|c|c|c|c|c|c|c|}
\hline
   \CE{ } & \CE{A} & \CE{B} & \CE{C} & \CE{D} & \CE{E} & \CE{F}  \\
\hline
 \CE{1} &  & 01/12/25 08:10 PM  &  &  &  &  \\
\hline
 \CE{2} &  & =YEAR(B1) &  &  &  &  \\
\end{tabular}
\end{table}

\begin{verbatim}
   2025
\end{verbatim}





\subsection{MONTH}
\begin{table}[H]
\centering
\begin{tabular}{|c|c|c|c|c|c|c|}
\hline
   \CE{ } & \CE{A} & \CE{B} & \CE{C} & \CE{D} & \CE{E} & \CE{F}  \\
\hline
 \CE{1} &  & 01/12/25 08:10 PM  &  &  &  &  \\
\hline
 \CE{2} &  & =MONTH(B1) &  &  &  &  \\
\end{tabular}
\end{table}

\begin{verbatim}
   12
\end{verbatim}





\subsection{DAY}
\begin{table}[H]
\centering
\begin{tabular}{|c|c|c|c|c|c|c|}
\hline
   \CE{ } & \CE{A} & \CE{B} & \CE{C} & \CE{D} & \CE{E} & \CE{F}  \\
\hline
 \CE{1} &  & 01/12/25 08:10 PM  &  &  &  &  \\
\hline
 \CE{2} &  & =DAY(B1) &  &  &  &  \\
\end{tabular}
\end{table}

\begin{verbatim}
   1
\end{verbatim}





\subsection{HOUR}
\begin{table}[H]
\centering
\begin{tabular}{|c|c|c|c|c|c|c|}
\hline
   \CE{ } & \CE{A} & \CE{B} & \CE{C} & \CE{D} & \CE{E} & \CE{F}  \\
\hline
 \CE{1} &  & 01/12/25 08:10 PM  &  &  &  &  \\
\hline
 \CE{2} &  & =HOUR(B1) &  &  &  &  \\
\end{tabular}
\end{table}

\begin{verbatim}
   20
\end{verbatim}





\subsection{MINUTE}
\begin{table}[H]
\centering
\begin{tabular}{|c|c|c|c|c|c|c|}
\hline
   \CE{ } & \CE{A} & \CE{B} & \CE{C} & \CE{D} & \CE{E} & \CE{F}  \\
\hline
 \CE{1} &  & 01/12/25 08:10 PM  &  &  &  &  \\
\hline
 \CE{2} &  & =MINUTE(B1) &  &  &  &  \\
\end{tabular}
\end{table}

\begin{verbatim}
   10
\end{verbatim}



\subsection{SECOND}
\begin{table}[H]
\centering
\begin{tabular}{|c|c|c|c|c|c|c|}
\hline
   \CE{ } & \CE{A} & \CE{B} & \CE{C} & \CE{D} & \CE{E} & \CE{F}  \\
\hline
 \CE{1} &  & 01/12/25 08:10 PM  &  &  &  &  \\
\hline
 \CE{2} &  & =SECOND(B1) &  &  &  &  \\
\end{tabular}
\end{table}

\begin{verbatim}
   50
\end{verbatim}




\section{String Functions}


\subsection{CHAR}
\begin{table}[H]
\centering
\begin{tabular}{|c|c|c|c|c|c|c|}
\hline
   \CE{ } & \CE{A} & \CE{B} & \CE{C} & \CE{D} & \CE{E} & \CE{F}  \\
\hline
 \CE{1} &  & =CHAR(65)  &  &  &  &  \\
\hline
 \CE{2} &  &  &  &  &  &  \\
\end{tabular}
\end{table}

\begin{verbatim}
   A
\end{verbatim}



\subsection{CODE}
\begin{table}[H]
\centering
\begin{tabular}{|c|c|c|c|c|c|c|}
\hline
   \CE{ } & \CE{A} & \CE{B} & \CE{C} & \CE{D} & \CE{E} & \CE{F}  \\
\hline
 \CE{1} &  & =CODE("B")  &  &  &  &  \\
\hline
 \CE{2} &  &  &  &  &  &  \\
\end{tabular}
\end{table}

\begin{verbatim}
   66
\end{verbatim}




\subsection{UNICODE}
\begin{table}[H]
\centering
\begin{tabular}{|c|c|c|c|c|c|c|}
\hline
   \CE{ } & \CE{A} & \CE{B} & \CE{C} & \CE{D} & \CE{E} & \CE{F}  \\
\hline
 \CE{1} &  & =UNICODE("\dSmiley")  &  &  &  &  \\
\hline
 \CE{2} &  &  &  &  &  &  \\
\end{tabular}
\end{table}

\begin{verbatim}
   9786
\end{verbatim}



\subsection{UNICHAR}
\begin{table}[H]
\centering
\begin{tabular}{|c|c|c|c|c|c|c|}
\hline
   \CE{ } & \CE{A} & \CE{B} & \CE{C} & \CE{D} & \CE{E} & \CE{F}  \\
\hline
 \CE{1} &  & =UNICHAR(9786)  &  &  &  &  \\
\hline
 \CE{2} &  &  &  &  &  &  \\
\end{tabular}
\end{table}


Output:   \dSmiley




\subsection{LOWER}
\begin{table}[H]
\centering
\begin{tabular}{|c|c|c|c|c|c|c|}
\hline
   \CE{ } & \CE{A} & \CE{B} & \CE{C} & \CE{D} & \CE{E} & \CE{F}  \\
\hline
 \CE{1} &  & istanBUL  &  &  &  &  \\
\hline
 \CE{2} &  & =LOWER(B1) &  &  &  &  \\
\end{tabular}
\end{table}

\begin{verbatim}
   istanbul
\end{verbatim}



\subsection{UPPER}
\begin{table}[H]
\centering
\begin{tabular}{|c|c|c|c|c|c|c|}
\hline
   \CE{ } & \CE{A} & \CE{B} & \CE{C} & \CE{D} & \CE{E} & \CE{F}  \\
\hline
 \CE{1} &  & istanBUL  &  &  &  &  \\
\hline
 \CE{2} &  & =UPPER(B1) &  &  &  &  \\
\end{tabular}
\end{table}

\begin{verbatim}
   İSTANBUL
\end{verbatim}




\subsection{CONCATENATE}
\begin{table}[H]
\centering
\begin{tabular}{|c|c|c|c|c|c|c|}
\hline
   \CE{ } & \CE{A} & \CE{B} & \CE{C} & \CE{D} & \CE{E} & \CE{F}  \\
\hline
 \CE{1} &  & İstanbul  &  University &  &  &  \\
\hline
 \CE{2} &  & =CONCATENATE(B1; C1) &  &  &  &  \\
\end{tabular}
\end{table}

\begin{verbatim}
 İstanbulUniversity
\end{verbatim}


\begin{table}[H]
\centering
\begin{tabular}{|c|c|c|c|c|c|c|}
\hline
   \CE{ } & \CE{A} & \CE{B} & \CE{C} & \CE{D} & \CE{E} & \CE{F}  \\
\hline
 \CE{1} &  & İstanbul  &  University &  &  &  \\
\hline
 \CE{2} &  & =CONCATENATE(B1;" "; C1) &  &  &  &  \\
\end{tabular}
\end{table}

\begin{verbatim}
 İstanbul University
\end{verbatim}




\subsection{LEFT}
\begin{table}[H]
\centering
\begin{tabular}{|c|c|c|c|c|c|c|}
\hline
   \CE{ } & \CE{A} & \CE{B} & \CE{C} & \CE{D} & \CE{E} & \CE{F}  \\
\hline
 \CE{1} &  & İstanbul  &  University &  &  &  \\
\hline
 \CE{2} &  & =LEFT(B1; 3) &  &  &  &  \\
\end{tabular}
\end{table}

\begin{verbatim}
 İst
\end{verbatim}





\subsection{RIGHT}
\begin{table}[H]
\centering
\begin{tabular}{|c|c|c|c|c|c|c|}
\hline
   \CE{ } & \CE{A} & \CE{B} & \CE{C} & \CE{D} & \CE{E} & \CE{F}  \\
\hline
 \CE{1} &  & İstanbul  &  University &  &  &  \\
\hline
 \CE{2} &  & =RIGHT(B1; 3) &  &  &  &  \\
\end{tabular}
\end{table}

\begin{verbatim}
 bul
\end{verbatim}




\subsection{MID}
\begin{table}[H]
\centering
\begin{tabular}{|c|c|c|c|c|c|c|}
\hline
   \CE{ } & \CE{A} & \CE{B} & \CE{C} & \CE{D} & \CE{E} & \CE{F}  \\
\hline
 \CE{1} &  & İstanbul  &  University &  &  &  \\
\hline
 \CE{2} &  & =MID(B1; 6; 3) &  &  &  &  \\
\hline 
 \CE{3} &  & =MID(C1; 5; 2) &  &  &  &  \\
\end{tabular}
\end{table}

\begin{verbatim}
 bul
 er
\end{verbatim}




\subsection{LEN}
\begin{table}[H]
\centering
\begin{tabular}{|c|c|c|c|c|c|c|}
\hline
   \CE{ } & \CE{A} & \CE{B} & \CE{C} & \CE{D} & \CE{E} & \CE{F}  \\
\hline
 \CE{1} &  & İstanbul  &  University &  &  &  \\
\hline
 \CE{2} &  & =LEN(B1) &  &  &  &  \\
\hline 
 \CE{3} &  & =LEN(C1) &  &  &  &  \\
\end{tabular}
\end{table}

\begin{verbatim}
   8
   10
\end{verbatim}




\subsection{ISTEXT}
\begin{table}[H]
\centering
\begin{tabular}{|c|c|c|c|c|c|c|}
\hline
   \CE{ } & \CE{A} & \CE{B} & \CE{C} & \CE{D} & \CE{E} & \CE{F}  \\
\hline
 \CE{1} &  & İstanbul  &  1453 &  &  &  \\
\hline
 \CE{2} &  & =ISTEXT(B1) &  &  &  &  \\
\hline 
 \CE{3} &  & =ISTEXT(C1) &  &  &  &  \\
\end{tabular}
\end{table}

\begin{verbatim}
   TRUE
   FALSE
\end{verbatim}






\subsection{ISNONTEXT}
\begin{table}[H]
\centering
\begin{tabular}{|c|c|c|c|c|c|c|}
\hline
   \CE{ } & \CE{A} & \CE{B} & \CE{C} & \CE{D} & \CE{E} & \CE{F}  \\
\hline
 \CE{1} &  & İstanbul  &  1453 &  &  &  \\
\hline
 \CE{2} &  & =ISNONTEXT(B1) &  &  &  &  \\
\hline 
 \CE{3} &  & =ISNONTEXT(C1) &  &  &  &  \\
\end{tabular}
\end{table}

\begin{verbatim}
   FALSE
   TRUE
\end{verbatim}


\subsection{REPLACE}
\begin{table}[H]
\centering
\begin{tabular}{|c|c|c|c|c|c|c|}
\hline
   \CE{ } & \CE{A} & \CE{B} & \CE{C} & \CE{D} & \CE{E} & \CE{F}  \\
\hline
 \CE{1} &  & İstanbul  &   &  &  &  \\
\hline
 \CE{2} &  & =REPLACE(A1; 4; 2; "*") &  &  &  &  \\
\hline 
 \CE{3} &  & =REPLACE(A1; 3; 5; "?") &  &  &  &  \\
\end{tabular}
\end{table}

\begin{verbatim}
   İs*bul
   İs?l
\end{verbatim}



\subsection{SEARCH}
\begin{table}[H]
\centering
\begin{tabular}{|c|c|c|c|c|c|c|}
\hline
   \CE{ } & \CE{A} & \CE{B} & \CE{C} & \CE{D} & \CE{E} & \CE{F}  \\
\hline
 \CE{1} &  & İstanbul  &   &  &  &  \\
\hline
 \CE{2} &  & =SEARCH("bul"; A1) &  &  &  &  \\
\hline 
 \CE{3} &  & =SEARCH("tan"; A1) &  &  &  &  \\
\hline 
 \CE{4} &  & =SEARCH("something"; A1) &  &  &  &  \\

\end{tabular}
\end{table}

\begin{verbatim}
   6
   3
   #VALUE!
\end{verbatim}






\subsection{VLOOKUP}

Basic Usage:

\begin{verbatim}
=VLOOKUP(lookup_value; table_array; col_index_num; [range_lookup])
\end{verbatim}

\noindent where 
\begin{itemize}
\item \texttt{lookup\_value} is the value to search for in the first column of the table\_array
\item \texttt{table\_array} is the range of cells that contains the data
\item \texttt{col\_index\_num} is the column number in the table\_array from which to retrieve the value
\item \texttt{range\_lookup} is a logical value that specifies whether to find an exact match (FALSE) or an approximate match (TRUE). If omitted, TRUE is assumed
\end{itemize}


\begin{table}[H]
\centering
\begin{tabular}{|c|c|c|c|c|c|c|}
\hline
   \CE{ } & \CE{A} & \CE{B} & \CE{C} & \CE{D} & \CE{E} & \CE{F}  \\
\hline
 \CE{1} & 2 &  Economics  &   &  &  &  \\
\hline
 \CE{2} & 3 &  Industrial Relations &  &  &  &  \\
\hline 
 \CE{3} & 5 & Econometrics &  &  &  &  \\
\hline 
 \CE{4} & 8 & Business Administration &  &  &  &  \\
\hline
   \CE{5} &  & =VLOOKUP(8; A1:B4; 2; FALSE) &  &  &  &  \\
\end{tabular}
\end{table}

\begin{verbatim}
   Business Administration
\end{verbatim}

Description of the example:

\begin{itemize}
\item The function looks for the value 8 in the first column of the range A1:B4
\item It finds the value 8 in cell A4
\item It then retrieves the value from the second column of the same row, which is "Business Administration"
\end{itemize}



\begin{table}[H]
\centering
\begin{tabular}{|c|c|c|c|c|c|c|}
\hline
   \CE{ } & \CE{A} & \CE{B} & \CE{C} & \CE{D} & \CE{E} & \CE{F}  \\
\hline
 \CE{1} & 101 &  Introduction to Economics  &  4 &  &  &  \\
\hline
 \CE{2} & 102 &  Information Technologies &  2  &  &  &  \\
\hline 
 \CE{3} & 103 &  Basic Statistics & 3 &  &  &  \\
\hline 
 \CE{4} & 104 &  Operations Research & 3 &  &  &  \\
\hline
   \CE{5} &  & =VLOOKUP(104; A1:C4; 3; FALSE) &  &  &  &  \\
\end{tabular}
\end{table}

The output: 

\begin{verbatim}
   3
\end{verbatim}

Why? Because the function looks for the value 104 in the first column of the range A1:C4,
finds it in cell A4, and retrieves the value from the third column of the same row, which is 3.


\section{Visual Basic for Applications (VBA)}

\subsection{\texttt{times} function}
\begin{minted}{basic}
' This function returns the product of two numbers
Function times(a, b)
    times = a * b
End Function
\end{minted}

Usage in Excel:
\begin{verbatim}
=times(5; 6)
\end{verbatim}




\subsection{\texttt{factorial} function}
\begin{minted}{basic}
Rem This function returns factorial of n
Function factorial(n)
    Dim result As Long
    Dim i As Long
    result = 1
    
    For i = 2 To n
        result = result * i
    Next

    factorial = result
End Function
\end{minted}

Usage in Excel:

\begin{verbatim}
=factorial(5)
\end{verbatim}



\subsection{\texttt{paraboladelta} function}
\begin{minted}{basic}
Rem This functions returns delta of a given parabola
Function paraboladelta(a As Double, b As Double, c As Double) As Double
    paraboladelta = b * b - 4 * a * c
End Function
\end{minted}

Usage in Excel:
\begin{verbatim}
=paraboladelta(1; -3; 2)
\end{verbatim}


\subsection{\texttt{numberofroots} function}
\begin{minted}{basic}
' This function returns
' -> 0, if delta < 0
' -> 1, if delta = 0
' -> 2, if delta > 0
Function numberofroots(a As Double, b As Double, c As Double) As Integer
    Dim delta As Double
    delta = b * b - 4 * a * c
    If delta < 0 Then
        numberofroots = 0
    ElseIf delta = 0 Then
        numberofroots = 1
    Else
        numberofroots = 2
    End If
End Function
\end{minted}

Usage in Excel:

\begin{verbatim}
=numberofroots(1; -3; 2)
\end{verbatim}



\subsection{\texttt{findminimum} function}
\begin{minted}{basic}
' This function returns the minimum of a given range
Function findminimum(r As Range) As Double
    mymin = 9999999
    For i = 1 To r.Count
        If r.Cells(i, 1) < mymin Then
            mymin = r.Cells(i, 1)
        End If
    Next
    findminimum = mymin
End Function
\end{minted}

Usage in Excel:

\begin{verbatim}
=findminimum(A1:A10)
\end{verbatim}


% Add the bibliography
\bibliographystyle{plain}
\bibliography{references}

\end{document}
